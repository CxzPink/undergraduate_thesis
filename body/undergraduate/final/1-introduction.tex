\cleardoublepage

\section{引言2000字}

\subsection{研究意义}
近年来,神经网络的成功应用促进了模式识别和数据挖掘等领域的研究。许多机器学习任务
,如目标检测[1],机器翻译[2]和语音识别[3],曾经非常依赖人工来提取特征信息,现在可
以通过各种端到端深度学习范式来完成,如卷积神经网络[4],循环神经网络[5]和自动编码
器[6]。深度学习在很多领域的成功归功于快速发展的计算资源(GPU)、大量可用于训练的
数据、以及从欧几里得数据(图像、文本、视频)中提取特征的能力。以图像数据为例,我们
可以将欧几里得空间中的图像表示为规则网络,卷积神经网络能够利用图像数据[7]的平移不变性、
局部连通性和合成性来提取特征,因此卷积神经网络可以提取能够应用于整个数据集的局部有效
特征信息。

虽然深度学习可以有效地捕捉欧几里得数据的特征,但越来越多的应用程序将数据表示为图数据。
对于一些从非欧几里得域生成的数据,如社会网络、经济网络、信息网络、流行病学网络、传感
器网络中的数据,人们处理和分析他们的需求正在日益增长。
具体来说,在电子商务中,基于图形的学习系统可以利用用户和产品之间的交互来做出高度准确
的推荐。在化学中,分子被建模为图形,它们的生物活性需要被识别以用于药物发现。在引
文网络中,论文是通过引文相互联系的,它们需要被分成不同的组。

然而,图数据的复杂性给现有的机器学习算法带来了巨大的挑战,这主要有以下原因。首先,由于图
可能是不规则的,一个图可能有一个可变大小的无序节点;其次,一个图的节点可能有不同数量的
邻居,导致一些重要的操作(如卷积)很容易在图像域计算,但很难应用到图域;此外,现有机器
学习算法的一个核心假设是实例(节点)是相互独立的,这个假设不再适用于图形数据,因为每个
实例(节点)通过各种类型的连接(如引用、交互)与其他实例(节点)相关联。

\subsection{研究现状}
为了图数据,在图信号处理领域,有学者提出了图信号的高效表示方法,并且建立了关于图
的移位、滤波、卷积、傅立叶变换、频谱分解等一系列概念;此外,有很多研究者
尝试使用图变换算子(GSO)的矩阵多项式,来设计可以分析图信号和设计的图滤波
器,这些图滤波器可以被用于信号预测、信号压缩、分类任务等应用。
\\
在深度学习领域,也正在有越来越多关于图神经网络(GNNs)的研究,学者们先
后提出了递归图神经网络(RecGNNs)、图卷积神经网络(ConvGNNs)、图自编码器
(GAEs)、时空图神经网络(STGNNs)等神经网络结构。其中,ConvGNNs 汲取了图信
号处理领域的思想,将卷积运算从网格数据推广到图数据。目前,ConvGNNs 已经
在节点级分类的半监督学习、图级分类的监督学习、用于图嵌入的无监督学习等任务里取得
了前所未有的效果;并且在其他复杂的GNNs 模型中,ConvGNNs 也发挥着核心作
用。
\subsubsection{递归图神经网络(RecGNNs)}
\subsubsection{图卷积神经网络(ConvGNNs)}
\subsubsection{图自编码器(GAEs)}
\subsubsection{时空图神经网络(STGNNs)}

\subsection{主要贡献}
\subsection{思路}


