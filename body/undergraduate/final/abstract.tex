\cleardoublepage{}
\begin{center}
    \bfseries \zihao{3} 摘要
\end{center}

随着计算能力的提升,机器学习、深度学习已经在图像、语音、文本等结构化数据的处理任务上取得了巨大成功,
但是用这些方法来处理非结构化数据时,表现却不够理想。近些年,有学者提出了专门用于非结构化数据处理的
图神经网络,但由于图数据的复杂性,图神经网络在表达能力、训练效率、适用性等方面依然有诸多缺陷。本文从
图信号处理的角度出发,充分结合频谱模型和空间模型的设计思想,创新性地提出了一种新的图卷积神经网络poly GAT。
实验结果表明,我们提出的神经网络有强大的表达能力,表达能力与当前最主流的图注意力神经网络(GAT)持平,
训练效率比它提高了至少一倍。

\noindent \textbf{关键词:} \ 深度学习,图卷积神经网络,图信号处理,图滤波器


\cleardoublepage{}
\begin{center}
    \bfseries \zihao{3} Abstract
\end{center}

Thanks to formidable computing power and massive data, machine learning and deep learning have achieved great 
success in the processing of structured data such as images, speech and text,but the 
performance of these methods in processing unstructured data is not satisfactory. 
In recent years, some researchors have proposed graph neural network(GNN) specially used for unstructured data processing. 
However, due to the complexity of graph data, graph neural network still has many defects in expression ability, 
training efficiency, applicability and other aspects. Inspired by graph signal processing, we propose poly GAT, 
a novel graph convolutional neural network,which fully combines the design ideas of spectrum model and space model. 
Our results indicate that the proposed neural network has strong expression ability equal to the best performing 
graph neural network (GAT), and the training efficiency is at least doubled.

\noindent \textbf{Index Terms:} \  deep learning, graph convolutional networks, graph signal processing, graph filter