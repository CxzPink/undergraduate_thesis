\cleardoublepage

\section{讨论与展望}

本文在前半部分总结了图卷积神经网络的两种主要理论及其设计方法,并且对比了他们在训练效率、
表达能力、图结构适用性上的差别。此外,还从图信号处理的角度,简述了设计图卷积神经网络的一
些思路,它认为图卷积神经网络不仅仅具有特征提取的功能,还有对输入信号中的误差进行滤波的作用,
它做出了可以将图卷积神经网络的一部分等效为低通滤波器的断论。

受到图信号处理研究的启发,我们提出了一种融合频谱理论和空间理论的图卷积神经网络poly GAT。
poly GAT在原有的传统GAT神经网络的结构基础上,增加了矩阵多项式的拟合。poly GAT在Cora等数
据集上有非常好的表现,它和传统GAT图卷积神经网络有几乎同样强大的表达能力,但相比传统GAT,
训练效率提高了一倍。

当然,我们工作还有很多需要在未来完善的地方。一方面,是关于poly GAT神经网络中多项式阶数的选取
问题,本文对阶数的选取做了相关的实验和思考,但是目前还没有一种方法可以自动地选择较优的
阶数。另一方面,poly GAT神经网络相对于GAT,虽然训练效率更好,但在表达能力上没有明显地提高,
所以未来如何提升poly GAT的表达能力,也是需要继续研究的方向。