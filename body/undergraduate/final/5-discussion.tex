\cleardoublepage

\section{讨论与展望}

本文总结了图卷积神经网络的两种主要理论及其设计方法,并且对比了他们在训练效率、
表达能力、图结构适用性上的差别。此外,我们还从图信号处理角度,简述了图卷积神经
网络分析和设计的近期研究进展,它认为图卷积神经网络不仅仅有特征提取的功能,还有
减少输入数据误差的作用,它做出了可以将图卷积神经网络的一部分等效为低通滤波器的断论。

受到图信号处理研究的启发,我们提出了一种融合频谱理论和空间理论的图卷积神经网络our GAT。
our GAT在原有的传统GAT神经网络的结构基础上,增加了矩阵多项式的拟合。Our GAT在Cora等数
据集上有非常好的表现,它和传统GAT图卷积神经网络有几乎同样强大的表达能力,但与传统GAT相比
,我们的训练效率比它提高了一倍。

当然,我们工作还有很多需要在未来完善的地方。一方面,是关于our GAT神经网络中多项式阶数的选取
问题,我们在本文对阶数的选取做了相关的实验和思考,但是目前还没有一种方法可以自动地选择较优的
阶数。另一方面,our GAT神经网络相对于GAT,虽然训练效率更好,但在表达能力上没有明显地提高,
所以未来如何提升our GAT的表达能力,也是我们需要继续研究的方向。