\cleardoublepage{}

\begin{center}
    \bfseries \zihao{3} 致谢
\end{center}

首先最想要感谢的,是指导我完成毕业论文的导师吴均峰(现为香港中文大学(深圳)的副教授),和共同指导我的南洋
理工大学的博士后丁克蜜(现为南方科技大学的助理教授)。是他们一步一步地引领我走进“图神经网络”这个研究
主题,阅读和讨论相关论文,并基于原有的方法提出了我们的创新,最终修改和定稿毕业论文,这些全都离不开他们无私的帮助。

接下去必须要感谢的,是浙江大学CAD\&CG国家重点实验室的黄劲教授和实验室的蒋静、孙典圣、郑一村学长。从数值
分析的课堂上领略到黄老师的精彩讲课,到听到黄老师鼓励我们要勇敢地探索学术上的核心问题,再到加入课题组并且
完成了一个属于自己的小项目,我在这个过程中提高了工程能力,完成了科研启蒙,并且激发了想要继续做研究的热情。
没有他们,我无法想象现在的自己会是怎样。

此外,我还想感谢班主任刘之涛老师、各位老师、父母、家人、女朋友、室友和同学们,是他们给予了我最充分的理解、
包容和精神上的支持。在本科阶段,生活烦恼、学业压力、前途迷惘就像影子一样时刻跟随着我,更别提好几次难忘的精神
上的滑铁卢。但正是他们的陪伴和鼓励,让我一次次勇敢地面对竞争和失败,一次次正视自我反思自我,一次次重拾信心。
也是在他们的见证下,我逐渐想清楚了竺可桢先生的两问,愈发坚定地走向未来。

最后,我特别想感谢我的博士生导师,香港科技大学的施凌教授,让我能有机会继续研究深奥有趣的控制理论。    